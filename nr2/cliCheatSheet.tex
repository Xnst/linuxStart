\documentclass[a4paper]{article}
%\usepackage[utf8]{inputenc}
\usepackage[swedish]{babel}
\usepackage{longtable,booktabs}
\begin{document}


\begin{longtable}{l|l}
  \toprule
  \textbf{kommando} & \textbf{beskrivning}\\
  \midrule
  \midrule
  \begin{minipage}[t]{0.2\textwidth} 
    \texttt{man}
  \end{minipage}
  &
  \begin{minipage}[t]{0.8\textwidth} 
    Visar dokumentationen f\"or det kommandot\\
    ex: \texttt{man ls}
  \end{minipage}
  \\
  
  \midrule
  \begin{minipage}[t]{0.2\textwidth} 
    \texttt{ls}
  \end{minipage}
  &
  \begin{minipage}[t]{0.7\textwidth} 
    Listar filer i katalogen, ibland \"aven underkatalogerna\\
    \begin{itemize}
    \item
      \texttt{ls} : alla synliga filer
    \item
      \texttt{ls -a} : alla filer, \"aven de ``osynliga''
    \item
      \texttt{ls -l} alla filer visade i l{\aa}ngt format
    \item
      \texttt{ls -lrth} : alla filer i l{\aa}ngt format{[}\texttt{l}{]} visade i omv\"and
      ordning{[}\texttt{r}{]} baserat p{\aa} senast \"andrad{[}\texttt{t}{]} och filstorlekar visat
      i k/M/G/T-byte{[}\texttt{h}{]}
    \item
      \texttt{ls *} : visar alla filer i katalogen, och om underkataloger finns visas
      \"aven filerna i dessa
    \end{itemize}
  \end{minipage}
  \\

  % ---    ----
  \midrule
  \begin{minipage}[t]{0.2\textwidth} 
    \texttt{cd}
  \end{minipage}
  &
  \begin{minipage}[t]{0.8\textwidth} 
    Byter katalog
    \begin{itemize}
      
    \item
      \texttt{cd} : byter till hemkatalogen
    \item
      \texttt{cd ~/} : byter till hemkatalogen
    \item
      \texttt{cd /} : byter till rooten
    \item
      \texttt{cd ../../} : byter till tv{\aa} steg n\"armre rooten
    \item
      \texttt{cd -} : byter till senaste bes\"okta katalogen
    \item
      \texttt{cd werther} : byter till katalogen \emph{werther}
    \end{itemize}
  \end{minipage}
  \\

  % ---    ----
  \midrule
  \begin{minipage}[t]{0.2\textwidth} 
    \texttt{cp}
  \end{minipage}
  &
  \begin{minipage}[t]{0.8\textwidth} 
kopierar filer

\begin{itemize}

\item
  \texttt{cp fil.txt ful.txt}
\item
  \texttt{cp -r dir1/ dirIgen}
\end{itemize}
  \end{minipage}
  \\
  % ---    ----
  \midrule
  \begin{minipage}[t]{0.2\textwidth} 
    \texttt{rm}
  \end{minipage}
  &
  \begin{minipage}[t]{0.8\textwidth} 
tar bort filen (ofta utan att fr{\aa}ga om det \"ar ok - en config sak)

\begin{itemize}

\item
  \texttt{rm fil.txt}
\item
  \texttt{rm -r dir1/}
\end{itemize}

  \end{minipage}
  \\
  % ---    ----
  \midrule
  \begin{minipage}[t]{0.2\textwidth} 
    \texttt{mkdir}
  \end{minipage}
  &
  \begin{minipage}[t]{0.8\textwidth} 
Skapar en katalog

\begin{itemize}

\item
  \texttt{mkdir werther} : skapar katalogen \emph{werther} d\"ar du befinner dig
\item
  \texttt{mkdir -p werther/lidanden} : skapar katalogen \emph{lidanden} i
  katalogen \emph{werther}, och om inte \emph{werther} finns s{\aa} den
  ocks{\aa}
\end{itemize}
  \end{minipage}
  \\
% ---    ----
  \midrule
  \begin{minipage}[t]{0.2\textwidth} 
    \texttt{sudo}
  \end{minipage}
  &
  \begin{minipage}[t]{0.8\textwidth} 
k\"or ett kommando som superuser (root)

\begin{itemize}

\item
  \texttt{sudo rm fil1.txt} : tar bort filen \"aven om du inte \"ar \"agera till den
\item
  \texttt{sudo rm -f fil1.txt} : ``dj\"avlar i helvete'' tar bort filen \"aven om du
  inte \"ar \"agera till den
\item
  \texttt{sudo -s} : startar en session som root
\end{itemize}
  \end{minipage}
  \\
  % ---    ----
  \midrule
  \begin{minipage}[t]{0.2\textwidth} 
    \texttt{history}
  \end{minipage}
  &
  \begin{minipage}[t]{0.8\textwidth} 
listar alla tidigare kommandon

t.ex:

\begin{verbatim}
history

2044  emacs outline.md&
2045  ls nr[0-9]
2046  ls
2047  ls *
2048  history 
\end{verbatim}

vill k\"ora ls igen s{\aa}:

\begin{verbatim}
!2046
\end{verbatim}

  \end{minipage}
  \\
  % ---    ----
  \midrule
  \begin{minipage}[t]{0.2\textwidth} 
    \texttt{find}
  \end{minipage}
  &
  \begin{minipage}[t]{0.8\textwidth} 
hittar filer baserat p{\aa} s\"okinfo

\begin{itemize}

\item
  find ./ -name ``\emph{.py" : listar alla pythonfiler(}.py) som finns i
  alla underkataloger d\"ar du befinner dig (./)
\end{itemize}

  \end{minipage}
  \\
  % ---    ----
  \midrule
  \begin{minipage}[t]{0.2\textwidth} 
    \texttt{grep}
  \end{minipage}
  &
  \begin{minipage}[t]{0.8\textwidth} 
s\"oker i en fil efter en textstr\"ang och skriver ut den raden

\begin{itemize}

\item
  \texttt{grep textstr cliCheatSheet.md}

  \begin{itemize}
  
  \item
    ger : - grep textstr cliCheatSheet.md
  \end{itemize}
\item
  \texttt{grep -i} : s\"oker i filen utan att vara case sensitive
\item
  \texttt{grep -r} : s\"oker igenom filerna i underkatalogerna ocks{\aa}
\item
  \texttt{grep -A5 -i} : s\"oker i filen/erna utan att vara case sensitive och
  skriver ut 5 rader fr{\aa}n d\"ar str\"angen finns
\end{itemize}

  \end{minipage}
  \\
  % ---    ----
  \midrule
  \begin{minipage}[t]{0.2\textwidth} 
    \texttt{locate}
  \end{minipage}
  &
  \begin{minipage}[t]{0.8\textwidth} 
s\"oker igenom registret \"over alla filer efter ett m\"onster

\begin{verbatim}
locate Cheat
\end{verbatim}

borde hitta denna fil om den finns p{\aa} datorn

indexet uppdateras med j\"amna mellanrum, beh\"ovs det uppdateras s{\aa} skriv

\begin{verbatim}
sudo updatedb
\end{verbatim}
  \end{minipage}
  \\
  % ---    ----
  \midrule
  \begin{minipage}[t]{0.2\textwidth} 
    \texttt{echo,\\ cat,\\ head,\\ tail,\\ less}
  \end{minipage}
  &
  \begin{minipage}[t]{0.8\textwidth} 
Skriver ut n{\aa}got

\begin{itemize}

\item
  \texttt{echo ``Hej!''} : skriver i term \emph{Hej!}
\item
  \texttt{echo \$PATH} : skriver i term v\"ardet p{\aa} parametern PATH
\item
  \texttt{cat text.txt} : skriver i term inneh{\aa}llet i text.txt
\item
  \texttt{head text.txt} : skriver i term de f\"orst 10 raderna i text.txt
\item
  \texttt{tail text.txt} : skriver i term de sista 10 raderna i text.txt
\item
  \texttt{tail -100 text.txt} : skriver i term de sista 100 raderna i text.txt
\item
  \texttt{tail -f text.txt} : skriver i term de sista 10 raderna i text.txt och
  v\"antar p{\aa} att nya rader skall komma f\"or att kunna visa dem.
\item
  \texttt{less text.txt} : \"oppnar text.txt s{\aa} att det g{\aa}r att scrolla. Avsluta
  med q.
\end{itemize}

  \end{minipage}
  \\
  % ---    ----
  \midrule
  \begin{minipage}[t]{0.2\textwidth} 
    \texttt{nano}
  \end{minipage}
  &
  \begin{minipage}[t]{0.8\textwidth} 
En enkel texteditor som k\"ors i terminalf\"onstret

\begin{verbatim}
nano text.txt
\end{verbatim}

spara med C-o, avsluta med C-x

  \end{minipage}
  \\
  % ---    ----
  \midrule
  \begin{minipage}[t]{0.2\textwidth} 
    \texttt{df}
  \end{minipage}
  &
  \begin{minipage}[t]{0.8\textwidth} 
listar alla monterade filsystem. Alla pseudofilsystem ocks{\aa}

\begin{itemize}

\item
  \texttt{df} : allt
\item
  \texttt{df -h} : allt presenterat i k/M/G/T-byte
\item
  \texttt{df -t ext4} : alla ext4 filsystem
\end{itemize}

  \end{minipage}
  \\
  % ---    ----
  \midrule
  \begin{minipage}[t]{0.2\textwidth} 
    \texttt{du}
  \end{minipage}
  &
  \begin{minipage}[t]{0.8\textwidth} 
visar storlek p{\aa} fil/katalog

\begin{itemize}

\item
  \texttt{du -skh *}: vsar storleken p{\aa} alla underkataloger i k/M/G/T-byte
\end{itemize}

  \end{minipage}
  \\
  % ---    ----
  \midrule
  \begin{minipage}[t]{0.2\textwidth} 
    \texttt{top}
  \end{minipage}
  &
  \begin{minipage}[t]{0.8\textwidth} 
visar systemanv\"andning f\"or processerna. avsluta med \emph{q}
  \end{minipage}
  \\
  % ---    ----
  \midrule
  \begin{minipage}[t]{0.2\textwidth} 
    \texttt{alias}
  \end{minipage}
  &
  \begin{minipage}[t]{0.8\textwidth} 
Skapar ett \emph{nytt} kommando. V\"aldigt bra att ha i sin .bashrc

\begin{verbatim}
alias lrt='ls -lrth'
\end{verbatim}
  \end{minipage}
  \\
  % ---    ----
  \midrule
  \begin{minipage}[t]{0.2\textwidth} 
    \texttt{\textbar{} (pipe)}
  \end{minipage}
  &
  \begin{minipage}[t]{0.8\textwidth} 
att pipa \"ar V\"ALDIGT anv\"andbart. t.ex.

\begin{verbatim}
locate config|grep openbox | grep home
\end{verbatim}

tar fram alla filer med config i namnet och ger hela s\"okv\"agen, sorterar
de som har openbox namnd i hela s\"okv\"agen och tar ut de som ocks{\aa} har
home i s\"okv\"agen
  \end{minipage}
  \\
  % ---    ----
  \midrule
  \begin{minipage}[t]{0.2\textwidth} 
    \texttt{xargs}
  \end{minipage}
  &
  \begin{minipage}[t]{0.8\textwidth} 
Anv\"ands ibland med pipe. Det som kommer ut ur kommandot innan (typ
filnamn) kan beh\"ova processas.

\begin{verbatim}
find ~/ -name *.py | xargs grep "import pandas"
\end{verbatim}

i vilka python-filer i min hemkatalog har jag anv\"ant pandas?
  \end{minipage}
  \\
  % ---    ----
  \midrule
  \begin{minipage}[t]{0.2\textwidth} 
    \texttt{export}
  \end{minipage}
  &
  \begin{minipage}[t]{0.8\textwidth} 
F\"or att s\"atta in variabel

\begin{verbatim}
export PYTHON_LIBRARY=$HOME/python
\end{verbatim}

  \end{minipage}
  \\
  % ---    ----
  \midrule
  \begin{minipage}[t]{0.2\textwidth} 
    \texttt{mount, umount}
  \end{minipage}
  &
  \begin{minipage}[t]{0.8\textwidth} 
monterar/avmonterar ett filsystem

\begin{itemize}

\item
  \texttt{sudo mount -o loop disk.iso mount-point-dir} : monterar iso-fil
\item
  \texttt{sudo umount mount-point-dir} : avmonterar
\item
  \texttt{sudo mount -t cifs //192.168.1.12/filer /mnt/serverFiler -o
  user=niclas,credentials=\\/etc/samba/smb-pw/win,domain=HOME,file\_mode=0777,\\dir\_mode=0777,uid=1000,gid=1000,iocharset=utf8}
\end{itemize}

  \end{minipage}
  \\
  % ---    ----
  \midrule
  \begin{minipage}[t]{0.2\textwidth} 
    \texttt{dmesg}
  \end{minipage}
  &
  \begin{minipage}[t]{0.8\textwidth} 
skriver ut hela h\"andelsef\"orloppen som k\"arnan k\"anner. Bra vid fels\"okning.
Pipa g\"arna in i less f\"or att kunna scrolla.

\begin{itemize}

\item
  \texttt{dmesg \textbar{} less}
\end{itemize}

  \end{minipage}
  \\
  % ---    ----
  \midrule
  \begin{minipage}[t]{0.2\textwidth} 
    \texttt{ip}
  \end{minipage}
  &
  \begin{minipage}[t]{0.8\textwidth} 
Anv\"ander oftast bara f\"or att se ip-adress : \texttt{ip a}
  \end{minipage}
  \\
  % ---    ----
  \midrule
  \begin{minipage}[t]{0.2\textwidth} 
    \texttt{screen\\eller\\tmux}
  \end{minipage}
  &
  \begin{minipage}[t]{0.8\textwidth}
    anv\"ands f\"or att starta cli-instanser som g{\aa}r att ha
    k\"orandes utan att vara inloggad.

    \texttt{screen -d -R} : starter eller kopplar upp emot instans igen

\end{minipage}
  \\
  % ---    ----
  \midrule
  \begin{minipage}[t]{0.2\textwidth} 
    \texttt{tar}
  \end{minipage}
  &
  \begin{minipage}[t]{0.8\textwidth} 
    Packar upp ett tar-arkiv.

    \begin{itemize}
    \item \texttt{tar tvf paket.tar.gz} :listar alla filer som finns i
    \item \texttt{tar xvf paket.tar.gz} :packar upp alla filer som
      finns i
    \end{itemize}
  \end{minipage}
  \\
  % ---    ----

    \midrule
  \begin{minipage}[t]{0.2\textwidth} 
    \texttt{chmod och chown}
  \end{minipage}
  &
  \begin{minipage}[t]{0.8\textwidth} 
    �ndrar r�ttigheter resp. �gare

    \begin{itemize}
    \item \texttt{chown a+x fil.py} : Ger alla (\texttt{a})
      exekveringsr�tt (\texttt{x})
    \item \texttt{chown user: fil.txt} : g�r \texttt{user} users
      default grupp �gare till filen
    \end{itemize}
  \end{minipage}
  \\
  % ---    ----
  \midrule
  \begin{minipage}[t]{0.2\textwidth} 
    \texttt{apt och dpkg}
  \end{minipage}
  &
  \begin{minipage}[t]{0.8\textwidth} 
Program installeras n\"astan alltid i form av paket. P{\aa} debian-baserade
distributioner anv\"ands .deb-filer. P{\aa} redhat-baserade anv\"ands
.rpm-filer. Dessa paket inneh{\aa}ller programmen som skall instlleras och
information om vilka andra paket som beh\"ovs. P{\aa} debian-baserade system
anv\"ands apt och dpkg.

\begin{itemize}

\item
  \texttt{dpkg} installerar ett paket.
\item
  \texttt{apt} anv\"ander dpkg och h{\aa}ller dessutom koll p{\aa} vilka andra
  paket som skall installeras.
\end{itemize}

\textbf{om en deb-filh\"amtats:}

\begin{itemize}

\item
  \texttt{sudo dpkg -i program-2.3.4.deb}
\item
  om det inte g{\aa}r bra s{\aa} beh\"ovs oftast fler paket installeras s{\aa} k\"or

  \begin{itemize}
  
  \item
    \texttt{sudo apt -f install}
  \item
    \texttt{sudo dpkg -i program-2.3.4.deb}
  \end{itemize}
\end{itemize}

\textbf{om repositories anv\"ands}

\begin{itemize}

\item
  \texttt{sudo apt install program}
\end{itemize}

\textbf{annat hanterande med apt}

\begin{itemize}
\item
  \texttt{sudo apt update} : H\"amtar en lista p{\aa} vilka de senaste versionerna \"ar
\item
  \texttt{sudo apt upgrade} : installerar de senaste versionerna fr{\aa}n
  repositories
\item
  \texttt{sudo apt full-upgrade} : installerar de senaste versionerna fr{\aa}n
  repositories och tar med nyare saker som uppgraderar systemet\ldots{}
\item
  \texttt{apt search \emph{program}} : kollar om och vilka paket som finns
\item
  \texttt{sudo apt autoremove} : tar bort paket som inte beh\"ovs
\end{itemize}
  \end{minipage}\\
  \bottomrule
  
\end{longtable}
  


\end{document}
